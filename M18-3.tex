% !TEX TS-program = pdflatex
% !TEX encoding = UTF-8 Unicode

% This is a simple template for a LaTeX document using the "article" class.
% See "book", "report", "letter" for other types of document.

\documentclass[11pt]{article} % use larger type; default would be 10pt

\usepackage[utf8]{inputenc} % set input encoding (not needed with XeLaTeX)

\usepackage{url}

\usepackage{tabularx}

\usepackage{footnote}
\makesavenoteenv{tabularx}

\usepackage{verbatimbox}

\usepackage{catchfilebetweentags}

\setcounter{secnumdepth}{0}% % Turns off numbering for sections
\setlength{\tabcolsep}{3pt} % sets how much space to have between columns in tables

\usepackage[default,osfigures,scale=0.95]{opensans} %% Alternatively
%% use the option 'defaultsans' instead of 'default' to replace the
%% sans serif font only.
\usepackage[T1]{fontenc}

\usepackage{float} % here for H placement parameter

%%% Examples of Article customizations
% These packages are optional, depending whether you want the features they provide.
% See the LaTeX Companion or other references for full information.

%%% PAGE DIMENSIONS
\usepackage{geometry} % to change the page dimensions
\geometry{a4paper} % or letterpaper (US) or a5paper or....
\geometry{margin=1cm,bottom=2cm} % for example, change the margins to 2 inches all round
% \geometry{landscape} % set up the page for landscape
%   read geometry.pdf for detailed page layout information

\usepackage{graphicx} % support the \includegraphics command and options

% \usepackage[parfill]{parskip} % Activate to begin paragraphs with an empty line rather than an indent

%%% PACKAGES
\usepackage{booktabs} % for much better looking tables
\usepackage{array} % for better arrays (eg matrices) in maths
\usepackage{paralist} % very flexible & customisable lists (eg. enumerate/itemize, etc.)
\usepackage{verbatim} % adds environment for commenting out blocks of text & for better verbatim
\usepackage{subfig} % make it possible to include more than one captioned figure/table in a single float
% These packages are all incorporated in the memoir class to one degree or another...

%%% HEADERS & FOOTERS
\usepackage{fancyhdr} % This should be set AFTER setting up the page geometry
\pagestyle{fancy} % options: empty , plain , fancy
\renewcommand{\headrulewidth}{0pt} % customise the layout...
\lhead{}\chead{}\rhead{}
\lfoot{}\cfoot{\thepage}\rfoot{}

%%% SECTION TITLE APPEARANCE
\usepackage{sectsty}
%\allsectionsfont{\sffamily\mdseries\upshape} % (See the fntguide.pdf for font help)
% (This matches ConTeXt defaults)

%%% ToC (table of contents) APPEARANCE
\usepackage[nottoc,notlof,notlot]{tocbibind} % Put the bibliography in the ToC
\usepackage[titles,subfigure]{tocloft} % Alter the style of the Table of Contents

\renewcommand{\cftsecfont}{\rmfamily\mdseries\upshape}
\renewcommand{\cftsecpagefont}{\rmfamily\mdseries\upshape} % No bold!

%%% END Article customizations

%%% The "real" document content comes below...

\title{NET103 MARIE Assembly Coursework Feedback\\ 2017-2018 }
\author{Student: 
10072XXX \\ 
Group: M18-
3 }
%\date{} % Activate to display a given date or no date (if empty),
         % otherwise the current date is printed 

\begin{document}
\twocolumn
\maketitle

\ExecuteMetaData[file.tex]{am}

%%%%%%%%%%%% FIRST TABLE OF NORMAL DIVISIONS %%%%%%%%%%%%%%%%%
\noindent
\begin{table}[H]
\small
%\caption{Division\label{tab:div}}
\addvbuffer[0pt 0pt]{\begin{tabularx}{\columnwidth}{c c c c c r }
  \hline
\multicolumn{2}{c}{Inputs} & \multicolumn{2}{X}{Expected output} &  \multicolumn{2}{X}{Your output}  \\
	Dividend & Divisor  & Quotient & Rest &  Quotient & Rest    \\
  \hline 
  	0 		& 1000  	& 0 		&  0 		& 1 		& -1000	\\ 
	2 		& 1 		& 2 		&  0 		& 2 		& 0		\\
	145 		& 12  	& 12 		&  1 		& 12 		& 1		\\
	233 		& 123  	& 1 		& 110 	&  1 		& 110	\\
	32767 	& 500  	& 65 		& 267  	& 65 		& 267 	\\
  \hline
\end{tabularx}}
\end{table}
%%%error handling
\ExecuteMetaData[file.tex]{diver}


%%%%%%%%%%%% TABLE OF ERROR DIVISIONS %%%%%%%%%%%%%%%%%
\noindent
\begin{table}[H]
\scriptsize
%\caption{Division: advanced requirements\label{tab:diverr}}
\addvbuffer[0pt 0pt]{\begin{tabularx}{\columnwidth}{l c X c X r }
  \hline
\multicolumn{2}{c}{Inputs} & \multicolumn{2}{c}{Expected output} 											&  \multicolumn{2}{r}{Your output}  \\
	Dividend	 & Divisor  	& Quotient & Rest 															&  Quotient 	& Rest    		\\
  \hline 
	0 		& 0  		&  \multicolumn{2}{l}{Err.\footnote[1]{a}} 											& 0 & 0 \\%???.\footnote[2]{a} 			\\
	1 		& 0  		&  \multicolumn{2}{l}{Err.\footnote[1]{a}} 											& 0 & 1 \\%???.\footnote[2]{a} 			\\
	-11 		& 2  		&  \multicolumn{2}{l}{Err.\footnote[1]{a} or -5 rest -1 or -6 rest 1\footnote[3]{a} } 			& 0 & -11  		\\
	111 		& -3  	&  \multicolumn{2}{l}{Err.\footnote[1]{a} or -37 rest 0 }  								& 0 & 111 			\\
	-12677	& -137  	&  \multicolumn{2}{p{3cm}}{Err.\footnote[1]{a} or -93 rest 64 or -92 rest -73\footnote[3]{a}}	& 0 & -12677 \\%???.\footnote[2]{a} 			\\
  \hline

\end{tabularx}}
\ExecuteMetaData[file.tex]{diverfootnotes}


\end{table}
  \normalsize
  
\ExecuteMetaData[file.tex]{assespri}

%%%%%%%%%%%% TABLE OF PRIMES basic%%%%%%%%%%%%%%%%%
\begin{table}[H]
\small
\centering
%\caption{Primes\label{tab:pri}}
\addvbuffer[0pt 0pt]{\begin{tabularx}{\columnwidth}{  X X r  }
\hline
Input\footnote[1] & Expected Output & Your output \\
\hline
1 	& 0 & ???\footnote[2]{} \\
2 	& 1 & 1 \\
3 	& 1 & 1 \\
4 	& 0 & 0 \\
5 	& 1 & 1 \\
6 	& 0 & 0 \\
7 	& 1 & 1 \\
89 	& 1 & 1 \\
201 	& 0 & 0 \\
577 	& 1 & 1 \\
649 	& 0 & 0 \\


\hline
\end{tabularx}}

\ExecuteMetaData[file.tex]{prifootnotes}

\end{table}

\ExecuteMetaData[file.tex]{assesprierr}

%%%%%%%%%%%% TABLE OF PRIMES errors%%%%%%%%%%%%%%%%%
\begin{table}[H]
\scriptsize
\centering
%\caption{Primes: advanced requirements}
\addvbuffer[12pt 12pt]{\begin{tabularx}{\columnwidth}{  l X X p{2.5cm}  }
\hline
Input\footnote[1]{} 	& Expected Output 		& Your output 	& Num. of operations \\
\hline
-1 				& 0 or Err.\footnote[2]{} 	& 1 			& 147485\\
0 				& 0 					& 1 			& 294932\\
899 				& 0 					& 0  			& 47497\\
983 				& 1 					& 1  			& 88787\\
4817 			& 1 					& 1  			& 504371\\
4819 			& 0 					& 0 			& 288571\\
32749 			& 1 					& ???\footnote[3]{}			&1000000+\\
32761 			& 0 					& ???\footnote[3]{}			&1000000+\\
32767 			& 0 					& ???\footnote[3]{}  			&1000000+\\
64507 (-1029)		& 0 or Err.\footnote[2]{} 	& 1			& 317\\
65521 (-15)		& 1 or Err.\footnote[2]{} 	& 1			& 18461\\
\hline
\end{tabularx}}
\ExecuteMetaData[file.tex]{prierrfootnotes}

\end{table}


\ExecuteMetaData[file.tex]{assesmetcrit}

\section{Results:}

Accuracy of Results: 16/40 \\
Functionality and Efficiency: = 11/30 \\ 
Documentation and commenting: 17/30 \\ 
Overall Mark: 44/100 \\

\section{Comments:}

\ExecuteMetaData[file.tex]{cgavg}
ACCURACY:
\ExecuteMetaData[file.tex]{caa3}
ELABORATE ACCURACY:
\ExecuteMetaData[file.tex]{cae3}
FUNCTIONALITY, RANGE:
\ExecuteMetaData[file.tex]{cfn}
\ExecuteMetaData[file.tex]{cfs}
\ExecuteMetaData[file.tex]{cfnn3}
\ExecuteMetaData[file.tex]{cfie}\\
\\ 
DOCUMENTATION:
\ExecuteMetaData[file.tex]{cdn} 
%However the misspelled words such as ``Psuedocode'' and ``Psuedocde'' [sic] do not give a good impression about it. \\
\\
RECOMMENDATIONS:
\ExecuteMetaData[file.tex]{crH}
\begin{itemize}
%\item \ExecuteMetaData[file.tex]{cra}
\item \ExecuteMetaData[file.tex]{cra3}
\item \ExecuteMetaData[file.tex]{crb3}
%\item \ExecuteMetaData[file.tex]{crc}
\item \ExecuteMetaData[file.tex]{crc3}
\item \ExecuteMetaData[file.tex]{crc3a}
%%\item \ExecuteMetaData[file.tex]{crd}
\item \ExecuteMetaData[file.tex]{cre}
%\item \ExecuteMetaData[file.tex]{crf}
%\item \ExecuteMetaData[file.tex]{crg}
%\item \ExecuteMetaData[file.tex]{crh}
\item \ExecuteMetaData[file.tex]{cri}
%\item \ExecuteMetaData[file.tex]{crj}
\item \ExecuteMetaData[file.tex]{crj3}
%\item \ExecuteMetaData[file.tex]{crk}
%\item \ExecuteMetaData[file.tex]{crl}
%\item \ExecuteMetaData[file.tex]{crm}
%\item \ExecuteMetaData[file.tex]{crn}
\end{itemize}
OVERALL: 
\ExecuteMetaData[file.tex]{co3}

\end{document}
